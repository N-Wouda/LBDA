\documentclass[12pt, english]{article}

\usepackage{listings}
\usepackage{xcolor}
\usepackage{hyperref}

\lstset{
	language=C++,
	basicstyle=\ttfamily\footnotesize,
	frame=lines,
	keywordstyle=\color{blue},
	tabsize=4
}

\title{SMIPS v1.0 user manual}
\author{
	Niels van der Laan
	\and
	Ward Romeijnders
	\and
	Niels Wouda
}

\date{\today}

\begin{document}
	
\maketitle
\tableofcontents

\section{Introduction}
\label{sec:introduction}

TODO Ward en Niels, kunnen jullie dit wat invullen? Een stukje motivatie en wat van de wiskunde, zodat ik daar in de examples op kan terugvallen.

\section{Installation instructions}
\label{sec:installation_instructions}

TODO Can partially be copied from the repository

\section{Examples}
\label{sec:examples}

TODO some text
  
\subsection{Solving an SMPS problem}
\label{subsec:solving_smps_example}

Solving a problem specified as a triplet of SMPS files is rather easy. Listings \ref{listing:smps_example_c++} and \ref{listing:smps_example_python} provide some example code, solving a problem provided in the data directory of the Github repository.
\begin{lstlisting}[caption={Solving an SMPS problem in C++.},
                   label={listing:smps_example_c++}]
#include "deterministicequivalent.h"
#include "masterproblem.h"
#include "problemdata.h"
#include "cutfamilies/lpdual.h"

int main()
{
	auto problem = ProblemData::fromSmps("data/electric/LandS");
	
	// Solves the deterministic equivalent formulation.
	DeterministicEquivalent detEqv(problem);
	auto decisions = detEqv.solve();
	
	std::cout << *decisions << '\n';
	std::cout << detEqv.objective() << '\n';
	std::cout << detEqv.firstStageObjective() << '\n';
	std::cout << detEqv.secondStageObjective() << '\n';
	
	// This applies the L-shaped algorithm.
	LpDual cutFamily(problem);
	MasterProblem master(problem, 0);
	decisions = master.solveWith(cutFamily);
	
	std::cout << *decisions << '\n';
	std::cout << master.objective() << '\n';
	std::cout << master.firstStageObjective() << '\n';
	std::cout << master.secondStageObjective() << '\n';
}
\end{lstlisting}
\begin{lstlisting}[caption={Solving an SMPS problem in Python.},
                   label={listing:smps_example_python},
                   language={Python}]
from smipspy import (DeterministicEquivalent, MasterProblem, 
                     ProblemData)
from smipspy.cutfamilies import LpDual

problem = ProblemData.from_smps("data/electric/LandS")

# Solves the deterministic equivalent formulation.
det_eqv = DeterministicEquivalent(problem)
decisions = det_eqv.solve()

print(decisions)
print(det_eqv.objective())
print(det_eqv.first_stage_objective())
print(det_eqv.second_stage_objective())

# Applies the L-shaped algorithm.
cut_family = LpDual(problem)

master = MasterProblem(problem, lower_bound=0)
decisions = master.solve_with(cut_family)

print(decisions)
print(master.objective())
print(master.first_stage_objective())
print(master.second_stage_objective())
\end{lstlisting}
Our SMPS reader is rather strict and might reject reading your files if they do not adhere to the standard \textit{exactly}. If this happens to you, the repository contains a simple Python script you can point to the offending problem set. The script is given in \texttt{data/fix\_faulty\_smps.py}.

In the listing above, we first read the problem data contained in the ``LandS'' files. A call to \texttt{fromSmps}/\texttt{from\_smps} returns an instance of the class \texttt{ProblemData} (Section \ref{subsec:problem_class}). All solution methods take such a \texttt{ProblemData} instance. Thereafter, we apply two solution methods. First, we construct and solve the deterministic equivalent, and print its optimal first-stage decisions, and objective costs. Then, we solve the same problem with the L-shaped algorithm. We also specify a lower bound of zero: since this is the default we did not have to, but in general it is important not to forget to pass a lower (and possibly upper) bound when using the decomposition. Since the ``LandS'' problem is fully continuous, we apply regular LP-duality based cuts. Observe that the interface for obtaining decisions and objective values is similar for both the decomposition-based \texttt{MasterProblem}, and the direct solution of the \texttt{DeterministicEquivalent}.

\subsection{Specifying your own problem data}
\label{subsec:own_data_example}

The \texttt{ProblemData} class exposes a constructor, which can be used to specify a problem instance that is not available in SMPS format. This is a bit more involved as there is quite a bit of data to specify. The argument order is given in the reference manual, particularly in Listing \ref{listing:problem_data}.

TODO which problem to develop (math and motivation)

The problem is then specified in C++ and Python as given in Listings \ref{listing:custom_example_c++} and \ref{listing:custom_example_python}.

\begin{lstlisting}[caption={Specifying a custom problem in C++.},
label={listing:custom_example_c++}]
#include "problemdata.h"

int main()
{
	auto problem = ProblemData();
}
\end{lstlisting}
\begin{lstlisting}[caption={Specifying a custom problem in Python.},
label={listing:custom_example_python},
language={Python}]
from smipspy import ProblemData

problem = ProblemData()
\end{lstlisting}
Once the problem is specified, the solution method is the same as for the other examples, and omitted here for brevity.

\section{Reference manual}
\label{sec:reference_manual}

We present a short reference manual of the public methods on classes one might commonly use from the
SMIPS library. The manual details the C++ interface. The names are similar in the Python interface, but are written in their Pythonic equivalents (\textit{e.g.} \texttt{firstStageObjective} in the C++ code is accessed as \texttt{first\_stage\_objective} in the Python interface). Both interfaces expect similar arguments. For example code in both languages, we refer to Section \ref{sec:examples}.

Our code uses the Armadillo numerical library. Particularly, we often rely on the vector, matrix, and sparse matrix classes offered by Armadillo. Such an Armadillo class can be recognised as belonging to the \texttt{arma} namespace (\textit{e.g.} \texttt{arma::vec} for a real-valued column vector, \texttt{arma::mat} for a real-valued matrix, and \texttt{arma::sp\_mat} for a real-valued sparse matrix in compressed column format). The Armadillo library and documentation is available at \href{http://arma.sourceforge.net/}{their website}.

\subsection{The \texttt{ProblemData} class}
\label{subsec:problem_class}

The \texttt{ProblemData} class contains all data describing a problem instance: the matrices $A$, $T$, and $W$, variable lower- and upper-bounds, constraint senses, objective coefficients, and scenario-specific right-hand sides. All solution methods solve problems described by a \texttt{ProblemData} instance. The public interface is given in Listing \ref{listing:problem_data}, and will be discussed below.
\begin{lstlisting}[caption={Public interface of the \texttt{ProblemData} class.},
                   label={listing:problem_data}]
class ProblemData
{
public:
    ProblemData(arma::sp_mat Amat,
				arma::sp_mat Tmat,
				arma::sp_mat Wmat,
				arma::mat scenarios,
				arma::vec scenarioProbabilities,
				arma::Col<char> firstStageConstrSenses,
				arma::Col<char> secondStageConstrSenses,
				arma::Col<char> firstStageVarTypes,
				arma::Col<char> secondStageVarTypes,
				arma::vec firstStageCoeffs,
				arma::vec secondStageCoeffs,
				arma::vec firstStageLowerBound,
				arma::vec firstStageUpperBound,
				arma::vec secondStageLowerBound,
				arma::vec secondStageUpperBound,
				arma::vec firstStageRhs);

	static ProblemData fromSmps(std::string const &location);
	
	arma::Col<char> const &firstStageConstrSenses() const;
	
	arma::Col<char> const &secondStageConstrSenses() const;
	
	arma::Col<char> &firstStageVarTypes();
	
	arma::Col<char> const &firstStageVarTypes() const;
	
	arma::Col<char> const &secondStageVarTypes() const;
		
	arma::sp_mat const &Amat() const;
	
	arma::sp_mat &Amat();
	
	arma::sp_mat const &Tmat() const;
	
	arma::sp_mat const &Wmat() const;
	
	arma::vec &firstStageCoeffs();
	
	arma::vec const &firstStageCoeffs() const;
	
	arma::vec const &secondStageCoeffs() const;
	
	arma::vec const &firstStageRhs() const;
	
	arma::vec const &firstStageLowerBound() const;
	
	arma::vec const &firstStageUpperBound() const;
	
	arma::vec const &secondStageLowerBound() const;
	
	arma::vec const &secondStageUpperBound() const;
	
	size_t nScenarios() const;
	
	arma::mat const &scenarios() const;
	
	double scenarioProbability(size_t scenario) const;
};
\end{lstlisting}
There are two ways to instantiate the \texttt{ProblemData} class: either by supplying your own vectors and matrices to the constructor, or supplying the filesystem location of an SMPS triplet of files to \texttt{fromSmps}. If one of the three SMSP files does not exist, \texttt{fromSmps} throws a \texttt{std::runtime\_error}.

Most of the member functions are data getters, returning parts of the data that define a two-stage recourse problem. The functions \texttt{Amat()}, \texttt{Tmat()} and \texttt{Wmat()} return references to the (sparse) matrices $A$, $T$, and $W$, respectively. \texttt{nScenarios()} returns the number of scenarios, \texttt{scenarios()} a dense matrix of all scenario right-hand sides (one column for each scenario). Finally, \texttt{scenarioProbability} returns the occurrence probability of the given scenario (number). The use of the other member functions is hopefully clear based on their names.

\subsection{The \texttt{DeterministicEquivalent} class}
\label{subsec:det_eqv_class}

The \texttt{DeterministicEquivalent} class formulates the two-stage recourse problem as a single deterministic (possibly mixed-integer) linear program, and solves it directly. This is sometimes also called the \textit{extensive form}. The public interface is given in Listing \ref{listing:det_eqv}.
\begin{lstlisting}[caption={Public interface of the \texttt{DeterministicEquivalent} class.},
                   label={listing:det_eqv}]
class DeterministicEquivalent
{
public:
	DeterministicEquivalent(ProblemData const &problem);
	
	std::unique_ptr<arma::vec> solve(
		double timeLimit = arma::datum::inf);
	
	double firstStageObjective();
	
	double secondStageObjective();
	
	double objective() const;
};
\end{lstlisting}
Its constructor takes a \texttt{ProblemData} instance to solve (see Section \ref{subsec:problem_class}). After construction, one can call the \texttt{solve} method, with an optional time limit argument (in seconds). When such an argument is not passed, the time limit defaults to infinity. If the time limit is exceeded, or the solution status is not optimal, \texttt{solve} throws a \texttt{std::runtime\_error}. The return argument is a (unique) pointer to a
vector of optimal first-stage decisions.

After solving the deterministic equivalent, the \texttt{firstStageObjective}, \texttt{secondStageObjective}, and \texttt{objective} methods return the optimal first-stage costs, the expected cost-to-go in the second-stage, and the overall objective cost, respectively.

\subsection{The \texttt{CutFamily} classes}
\label{subsec:cut_families}

TODO text after we've standardised the math.

\subsection{The \texttt{MasterProblem} class}
\label{subsec:master_problem_class}

The \texttt{MasterProblem} class solves the two-stage recourse problem by means of decomposition. In particular, it formulates the first-stage master problem, and for each second-stage scenario a suitable sub problem. It then applies a family of cutting planes (Section \ref{subsec:cut_families}) to iteratively approach an optimal solution. The public interface for this class is given in Listing \ref{listing:master_problem}.
\begin{lstlisting}[caption={Public interface of the \texttt{MasterProblem} class.}, 
                   label={listing:master_problem}]
class MasterProblem
{
public:
	MasterProblem(ProblemData &problem,
				  double lowerBound = 0.,
				  double upperBound = arma::datum::inf);
	
	void addCut(CutFamily::Cut &cut);
	
	std::unique_ptr<arma::vec> solveWith(CutFamily &cutFamily,
	                                     double tol = 1e-4);
	
	double firstStageObjective() const;
	
	double secondStageObjective() const;
	
	double objective() const;
};
\end{lstlisting}
Its constructor takes the familiar \texttt{ProblemData} instance, and two optimal arguments providing a lower- and upper-bound on $\theta$. If these are not passed, it is assumed that $0 \le \theta \le \infty$. After construction, one can call the \texttt{solveWith} method, with an appropriate family of cutting planes as an argument. This cut family is then used to derive optimality cuts solving the problem. An optional tolerance argument, \texttt{tol}, can be passed, which specifies the tolerance that should be used when evaluating the optimality of a current first-stage solution. The default tolerance is $0.0001$. If the master problem is ever infeasible, a \texttt{std::runtime\_error} is thrown. The return argument is a (unique) pointer to a vector of near-optimal first-stage decisions.

Once the master problem has been solved, the \texttt{firstStageObjective}, \texttt{secondStageObjective}, and \texttt{objective} methods return the optimal first-stage costs, the expected cost-to-go in the second-stage, and the overall objective cost, respectively.


	
\end{document}